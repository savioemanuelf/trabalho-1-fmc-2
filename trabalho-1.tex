\documentclass{article}
\usepackage{amssymb}
\usepackage{graphicx} 
\usepackage{amsmath}
\usepackage{amsfonts}

\title{Lista Fundamentos Matemáticos da Computação II - Respostas}

\begin{document}

\maketitle

\section{Seção Múltipla Escolha}

\subsection{Questão M1} 
Como $A = \{1, 3, 4\}$ [1] e $B = \{ x \in \mathbb{R} \ x^2 - 7x + 12 = 0\}$ [2], encontraremos os elementos de $B$ através da Relação de Girard de 2º, conhecida como Soma e Produto \\

Note que $a = 1$, $b = - 7$, $c = 12$, $c/a = 12$ e $-b/a = 7$. Assim, por Soma e Produto,
\[
\begin{aligned}
& x^2 - 7x + 12 = 0 \implies x \cdot x' = 12 \\
&x^2 - 7x + 12 = 0 \implies x + x' = 7 \\  
\end{aligned}
\]

Logo, os valores de $x$ e $x'$ seriam $3$ e $4$ [3]. Logo, $B = \{3, 4\}$ [4]. \\
Considerando as alternativas dadas pelo enunciado e o que foi afirmado em [1] e [4], a alternativa correta é a letra C, que afirma que $B \subset A$, visto que o conjunto $\{3, 4\} \subset \{1, 3, 4\}$ ($B$ é um subconjunto de $A$).


\subsection{Questão M2}
Considerando as seguintes proposições, analisemos seus valores lógicos 
\[
\begin{aligned}
& \text{1. } 0 \in \emptyset \\
& \text{2. } \emptyset \in \{0\} \\
& \text{3. } \{0\} \subset \emptyset\\
& \text{4. } \emptyset \subset \{0\} \\
& \text{5. } \{0\} \in \{0\} \\
& \text{6. } \{0\} \subset \{0\} \\
& \text{7. } \{\emptyset\} \subseteq \{\emptyset\}
\end{aligned}
\]

\[
\begin{aligned}
& \text{1. } 0 \in \emptyset &&\text{F, por definição, o $\emptyset$ não possui elementos.} \\
& \text{2. } \emptyset \in \{0\} &&\text{F, o $\emptyset$ é um subconjunto de todo conjunto, não elemento.}\\
& \text{3. } \{0\} \subset \emptyset &&\text{F, por definição, o $\emptyset$ não possui subconjuntos além de $\emptyset$.} \\
& \text{4. } \emptyset \subset \{0\} &&\text{V, por definição, o $\emptyset$ é um subconjunto de qualquer conjunto.} \\
& \text{5. } \{0\} \in \{0\} &&\text{F, o \{0\} é um subconjunto de \{0\}, mas não um elemento.} \\
& \text{6. } \{0\} \subset \{0\} &&\text{V, por definição, um dos subconjuntos de qualquer conjunto é ele mesmo.} \\
& \text{7. } \{\emptyset\} \subseteq \{\emptyset\} &&\text{V, o conjunto com o conjunto vazio é um subconjunto dele mesmo.} \\
\end{aligned}
\]
Logo, considerando as alternativas dadas pelo enunciado, a resposta é letra C, FFFVFVV.

\subsection{Questão M3}
% SUA RESPOSTA DA M3
\subsection{Questão M4}
Note que os conjuntos \( A \) e \( B \) são definidos como:

\[
A = \{x \in \mathbb{R} \mid 3 < x < 2\} \quad \text{e} \quad B = \{x \in \mathbb{Z} \mid 2 < x < 3\}.
\]

**Conjunto \( A \):** O intervalo \( 3 < x < 2 \) não possui elementos em \(\mathbb{R}\), pois não é possível \( x \) satisfazer \( 3 < x < 2 \). Assim, \( A = \emptyset \) e \(|A| = 0\).

**Conjunto \( B \):** Os números inteiros que satisfazem \( 2 < x < 3 \) são inexistentes, pois não há nenhum inteiro no intervalo \( 2 < x < 3 \). Portanto, \( B = \emptyset \) e \(|B| = 0\).

A soma das cardinalidades dos conjuntos é:

\[
|A| + |B| = 0 + 0 = 0.
\]

Logo, a resposta correta é a alternativa \(\mathbf{A}\).

\subsection{Questão M5}
% SUA RESPOSTA DA M5

\vspace{0.5em}
\hrule
\vspace{0.5em}

\section{Seção Discursiva}

\subsection{Questão D1}
Como o conjunto $A$ é definido por $A = \{ x \in \mathbb{N^*} | x \leq 10 \}$, isto é,
$A = \{1, 2, 3, 4, 5, 6, 7, 8, 9, 10\}$ e $P = \{\{1,2,5,6,7\},\{3\}, \{4, 8, 10\}, \{9\}\}$, podemos verificar se a união dos subconjuntos de $P$ resulta no conjunto $A$. Veja que

\[
\bigcup P = \{1, 2, 3, 4, 5, 6, 7, 8, 9, 10\}
\]
Essa união representa exatamente o mesmo que o conjunto $A$. Como os subconjuntos de $P$ são mutuamente disjuntos e todos contém ao menos um elemento, é possível afirmar que $P$ é uma partição de $A$.
\subsection{Questão D2}
% SUA RESPOSTA DA D2
\subsection{Questão D3}
% SUA RESPOSTA DA D3

\end{document}
